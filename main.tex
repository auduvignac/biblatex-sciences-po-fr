\documentclass{article}

% Langue et typographie
\usepackage[french]{babel}
\usepackage{csquotes}
\usepackage{fontspec} % XeLaTeX
\usepackage{geometry}
\geometry{a4paper, margin=1in}

% Mathématiques et tableaux
\usepackage{amsmath}
\usepackage{booktabs}

% Commandes avancées
\usepackage{xparse}

% Hyperliens
\usepackage{hyperref}

% --------------------------------------------------------------------
% Flag pour afficher toute la bibliographie (mode demonstration)
% --------------------------------------------------------------------
% Activation via ligne de commande :
% xelatex "\def\demobib{1}\documentclass{article}
\usepackage[utf8]{inputenc}
\usepackage[T1]{fontenc}
\usepackage[french]{babel} % Charger babel pour la typographie française générale
\usepackage{csquotes}
\usepackage{hyperref}

% Configuration de biblatex (activation d'ibid./op. cit. pour les notes)
\usepackage[
    style=authoryear, % Base pour la bibliographie
    citestyle=verbose-trad1, % Style de notes avec ibid./op. cit.
    backend=biber,    
    hyperref=true,
    pagetracker=true, % Assure le suivi des pages (pour ibid.)
    doi=false,        
    url=false,
    autolang=other % Empêche biblatex d'adapter la langue via babel (utile pour ibid./op. cit.)
]{biblatex}

% Fichiers bibliographiques
\addbibresource{bibliographies/livres.bib}
\addbibresource{bibliographies/rapports.bib}

% --- GESTION DU NOMBRE D'AUTEURS (et al.) ---
\AtBeginDocument{%
    % Limite à 3 auteurs avant de mettre et al.
    \setcounter{maxnames}{3}
    \setcounter{minnames}{1}
}
% Séparateurs entre auteurs : virgule + espace
\DeclareDelimFormat{multinamedelim}{\addcomma\space}
\DeclareDelimFormat{finalnamedelim}{\addcomma\space}

% --- FORMATTAGE DES NOMS : NOM, Prénom. (avec majuscules) ---
\DeclareNameFormat{labelname}{%
  % Même format pour tous les auteurs : NOM, Prénom.
  \nameparts{#1}%
  \ifdef\namepartfamily{%
    \ifblank{\namepartgiven}
      {\namepartfamily}% cas institution : respecter la casse
      {\MakeUppercase{\namepartfamily}}%
  }{}%
  \ifblank{\namepartgiven}{}{\addcomma\space\namepartgiven}%
  \ifnumless{\value{listcount}}{\value{liststop}}{\addcomma\space}{}%
}
% Utiliser ce format pour tous les auteurs (notes + bibliographie)
\DeclareNameAlias{author}{labelname}

% --- FORMATTAGE DES CHAMPS STANDARDS (pas d'italique, ponctuation) ---
\DeclareFieldFormat{title}{#1\adddot} 
\DeclareFieldFormat{edition}{#1\adddot}

% --- DRIVER POUR LIVRE (@book) ---
% Format: AUTEUR, Prénom. Titre. Edition. Lieu de publication : éditeur, année. Pagination.
\DeclareBibliographyDriver{book}{%
  \usebibmacro{bibindex}%
  \usebibmacro{begentry}%
  \printnames{author}%
  \newunit\newblock%
  \printfield{title}% 
  \newunit\newblock%
  \printfield{edition}% 
  \newunit\newblock%
  \printlist{location}% 
  \addcolon\space%
  \printlist{publisher}% 
  \addcomma\space%
  \printfield{year}\adddot%
  \addspace%
  \printfield{pages}%
  \usebibmacro{finentry}%
}

% --- DRIVER POUR RAPPORT (@report) ---
% Format: Auteur/Institution. Titre. Rapport <numéro>. Institution, année. Série.
\DeclareBibliographyDriver{report}{%
  \usebibmacro{bibindex}%
  \usebibmacro{begentry}%
  \renewcommand*{\newunitpunct}{\adddot\space}%
  \iffieldundef{shortinstitution}
    {\ifnameundef{author}{\printlist{institution}}{\printnames{author}}}
    {\printfield{shortinstitution}}%
  \setunit{\addperiod\space}%
  \printfield{title}%
  \setunit{\addperiod\space}%
  \printfield{type}\addspace\printfield{number}%
  \setunit{\addperiod\space}%
  \iffieldundef{organization}{\printlist{institution}}{\printlist{organization}}%
  \addcomma\space%
  \printfield{year}%
  \setunit{\space}%
  \printtext{\mkbibparens{\printfield{series}}}%
  \usebibmacro{finentry}%
}


\begin{document}

\section*{Texte d'exemple}

\subsection*{Citations}
% Les citations d'article ont été retirées pour se concentrer uniquement sur les livres
\footcite{Boursin:1991} % 1. Boursin (Première citation complète)
\footcite{Fitoussi:1995} % 2. Fitoussi (Première citation complète)
\footcite{Fitoussi:1995} % 3. Fitoussi (Devrait être ibid.)
\footcite[][150]{Fitoussi:1995} % 4. Fitoussi, p. 150 (Devrait être ibid., p. 150)
\footcite{Vernant:2000} % 5. Vernant (Citation complète)
\footcite{Fitoussi:1995} % 6. Fitoussi (Devrait être op. cit. / loc. cit.)
\footcite{Hermet:2000} % 7. Hermet et al. (doit produire et al.)
\footcite{Lazar:1991} % 8. Lazar et al. (citation complète)
\footcite{Insee:2020} % 9. Rapport Insee (rapport numéroté)
\footcite{Insee:2020} % 10. Rapport Insee (rapport numéroté, Devrait être ibid.)


\newpage
\printbibliography[heading=bibintoc, title=Bibliographie]

\end{document}
"

% --------------------------------------------------------------------
% Configuration biblatex (style Sciences Po)
% --------------------------------------------------------------------
\usepackage[
    style=sciencespo,
    backend=biber,
    hyperref=true,
    loccittracker=true, % Autorise loc. cit. (désactivé par défaut en pratique)
    pagetracker=true,   % Active ibid./op. cit.
    doi=false,
    url=true,
    urldate=long,
    autolang=other      % Maintient les locutions latines
]{biblatex}

% Fichiers bibliographiques
\addbibresource{bibliographies/articles.bib}
\addbibresource{bibliographies/livres.bib}
\addbibresource{bibliographies/theses.bib}
\addbibresource{bibliographies/films.bib}
\addbibresource{bibliographies/rapports.bib}
\addbibresource{bibliographies/online.bib}

% --------------------------------------------------------------------
% Commandes pédagogiques pour les exemples de citation
% --------------------------------------------------------------------

% Commande standard : partage l'historique global
% (utile pour tester ibid./op. cit.)
\NewDocumentCommand{\footciteexample}{ O{} O{} m }{%
  \footnote{%
    \fullcite[#1][#2]{#3}%
  }%
}

% Variante locale : isole chaque citation dans un refsection
% pour garantir des exemples sans ibid./op. cit.
\NewDocumentCommand{\footciteexamplelocal}{ O{} O{} m }{%
  \begin{refsection}%
    \footnote{%
      \fullcite[#1][#2]{#3}%
    }%
  \end{refsection}%
}

% Environnement standardisé pour les rendus d'exemples
% (citations dans le corps du texte, sans interaction entre elles)
\newenvironment{bibexample}
  {\begin{refsection}\small}
  {\end{refsection}}

% --------------------------------------------------------------------
% Métadonnées du document
% --------------------------------------------------------------------
\title{Mode d'emploi du style bibliographique \texttt{sciencespo}}
\author{Aurélien Duvignac-Rosa}
\date{Basé sur le Mémorandum Officiel de Sciences Po}

\begin{document}
\maketitle
\tableofcontents

\section{Pourquoi un style \texttt{sciencespo} ?}

Ce style reprend les exigences du mémorandum Sciences Po\footnote{Voir \texttt{documentation/citer-sources-rediger-bibliographie-fr.pdf}.} et s'appuie sur trois fichiers :
\begin{itemize}
	\item \texttt{sciencespo.cbx} : gestion des citations en notes (ibid./op. cit., ajout du \texttt{shorttitle}).
	\item \texttt{sciencespo.bbx} : format de la bibliographie (noms en majuscules, ordre des éléments).
	\item \texttt{biblatex-dm.cfg} : déclaration des champs additionnels (mois textuel, plateforme vidéo, discipline, etc.).
\end{itemize}
Ils sont basés sur \texttt{biblatex} avec \texttt{biber} et couvrent tous les exemples du mémo : livres, rapports, chapitres, articles, thèses, films, vidéos en ligne, articles en ligne et sites web.

\section{Règles générales de citation (rappel du mémo)}

\subsection{Droit d'auteur}
Respectez la convention de Berne : guillemets pour les citations littérales, mention systématique de l'auteur et du titre, citation courte par rapport au texte original. La paraphrase sans référence est assimilée à du plagiat.

\subsection{Construire la bibliographie}
\begin{itemize}
	\item Indiquer auteur, titre, source d'origine (revue, ouvrage collectif), date et pagination.
	\item Classer la bibliographie par ordre alphabétique des auteurs (ou par thèmes puis auteurs).
\end{itemize}

\section{Installer et charger le style}

\subsection{Placer les fichiers}
\begin{enumerate}
	\item Copier \texttt{style/sciencespo.cbx} dans \texttt{\$TEXMFHOME/tex/latex/biblatex/cbx/}.
	\item Copier \texttt{style/sciencespo.bbx} dans \texttt{\$TEXMFHOME/tex/latex/biblatex/bbx/}.
	\item Copier \texttt{style/biblatex-dm.cfg} dans \texttt{\$TEXMFHOME/tex/latex/biblatex/}.
\end{enumerate}
Avec Docker/Makefile, ces montages sont faits automatiquement (voir \texttt{docker-compose.yml}).

\subsection{Charger \texttt{biblatex}}
\begin{verbatim}
\usepackage[
    style=sciencespo,
    backend=biber,
    hyperref=true,
    pagetracker=true,
    doi=false,
    url=true,
    urldate=long,
    autolang=other
]{biblatex}
\end{verbatim}
\textbf{Pourquoi ces options ?}
\begin{itemize}
	\item \texttt{pagetracker} active ibid./op. cit. en bas de page.
	\item \texttt{autolang=other} conserve les locutions latines.
	\item \texttt{doi=false} et \texttt{url=true} reflètent le mémo (URLs seulement pour les ressources en ligne).
\end{itemize}
Ajoutez ensuite vos fichiers \texttt{.bib} via \verb|\addbibresource|.

\section{Citer dans le texte : notes de bas de page}

\subsection{Commandes principales}
\begin{itemize}
	\item \verb|\footcite{clé}| : citation courte en note.
	\item \verb|\footcite[23]{clé}| : ajout d'une page.
	\item \verb|\footcite[Voir aussi][65]{clé}| : préfixe libre + page.
	\item \verb|\footfullcite{clé}| : citation complète directement en note (utile pour la première mention dans certains travaux).
	\item \verb|\fullcite{clé}| : citation complète dans le corps du texte.
\end{itemize}

\subsection{Gestion automatique des répétitions}
Le \texttt{.cbx} repose sur \texttt{verbose-trad1} : répétition immédiate $\rightarrow$ ibid.; répétition non immédiate $\rightarrow$ op. cit.; numéros de pages mis à jour automatiquement. Renseignez le champ \texttt{shorttitle} dans vos \texttt{.bib} pour lever toute ambiguïté d'auteur lors d'un op. cit.

\subsection{Exemple complet de notes}
\begin{enumerate}
	\item \texttt{\textbackslash footcite[][15]\{Fitoussi:1995\}} \footcite[15]{Fitoussi:1995}
	\item \texttt{\textbackslash footcite[][15]\{Fitoussi:1995\}} \footcite[15]{Fitoussi:1995}
	\item \texttt{\textbackslash footcite[150]\{Fitoussi:1995\}} \footcite[150]{Fitoussi:1995}
	\item \texttt{\textbackslash footcite[][5]\{Leca:2001\}} \footcite[5]{Leca:2001}
	\item \texttt{\textbackslash footcite[300]\{Fitoussi:1995\}} \footcite[300]{Fitoussi:1995}
	\item \texttt{\textbackslash footcite[Voir aussi][65]\{Vernant:2000\}} \footcite[Voir aussi][65]{Vernant:2000}
\end{enumerate}
\textbf{À tester (notes)} : rejouer cette séquence pour vérifier ibid./op. cit. et les pages ; comparer \verb|\footcite| (forme courte) et \verb|\footfullcite| (forme complète en note) sur une même clé ; ajouter un second titre d'un même auteur avec \texttt{shorttitle} pour observer son affichage avant op. cit.

\section{Champs spécifiques supportés par le style}

\begin{table}[h]
	\centering
	\begin{tabular}{@{}llp{8cm}@{}}
		\toprule
		Type     & Champ                                                   & Utilisation par \texttt{sciencespo}                                                                          \\
		\midrule
		Tous     & \texttt{shorttitle}                                     & Affiché avant op. cit. pour distinguer deux titres du même auteur.                                           \\
		Rapport  & \texttt{shortinstitution}                               & Acronyme prioritaire lorsque l'auteur est absent.                                                            \\
		Article  & \texttt{monthtext}                                      & Mois/saison libre (ex. \texttt{fév.-avr. 2001}) en lieu et place de \texttt{month+year}.                     \\
		Chapitre & \texttt{editortype}                                     & Mention \og dir.\fg{} après le nom de l'éditeur scientifique.                                                \\
		Thèse    & \texttt{discipline}                                     & Discipline affichée après le type de thèse.                                                                  \\
		Film     & \texttt{medium}, \texttt{extent}                        & Support et durée : \og [cassette vidéo]\fg{}, \og 1 cass. vidéo VHS, 52 min.\fg{}                            \\
		En ligne & \texttt{platform}                                       & Plateforme (Statista, YouTube...) lorsque \texttt{journaltitle} n'est pas présent.                           \\
		En ligne & \texttt{pubdate}, \texttt{updatedate}, \texttt{urldate} & Dates de publication, de mise à jour et de consultation (affichées entre crochets).                          \\
		En ligne & \texttt{entrysubtype}                                   & \texttt{video} $\rightarrow$ \og [vidéo en ligne]\fg{} ; \texttt{website} $\rightarrow$ \og [en ligne]\fg{}. \\
		\bottomrule
	\end{tabular}
\end{table}

\section{Fiches de saisie par type d'entrée}
Chaque entrée \texttt{.bib} ci-dessous correspond à un exemple du mémo Sciences Po et illustre le rendu \texttt{\textbackslash fullcite}.

\subsection{Principes méthodologiques, typologie des ouvrages et locutions latines}

\paragraph{Principes méthodologiques}

Les exemples de citation présentés dans cette documentation sont encapsulés
dans des environnements \texttt{refsection} afin d'éviter toute interaction
avec l'historique global des citations (notamment l'emploi automatique de
\emph{ibid.}, \emph{idem} ou \emph{op. cit.}). Cette encapsulation garantit
un rendu stable et reproductible des exemples, indépendamment du contexte
du document.

La commande \texttt{\textbackslash footciteexample} est réservée aux usages
pédagogiques illustratifs. Elle ne doit pas être utilisée dans le corps
analytique du rapport, où les citations doivent recourir exclusivement aux
commandes standard \texttt{\textbackslash cite},
\texttt{\textbackslash footcite} ou \texttt{\textbackslash fullcite},
selon le niveau de détail requis.

\paragraph{Typologie des ouvrages}

Les entrées de type \texttt{@book} couvrent l'ensemble des cas usuels rencontrés
dans les travaux en sciences humaines et sociales. La présente documentation
distingue les situations suivantes, sans modifier le type bibliographique,
les différences étant portées par les champs descriptifs de l'entrée :

\begin{itemize}
	\item \textbf{Livre (monographie) avec mention d'édition}, lorsque l'éditeur
	      précise explicitement une édition autre que la première
	      (champ \texttt{edition}) ;
	\item \textbf{Livre (monographie) sans mention d'édition explicite}, correspondant
	      au cas général des ouvrages publiés sans précision d'édition ;
	\item \textbf{Livre à auteurs multiples}, pour lequel l'abréviation automatique
	      (\emph{et al.}) peut être activée à l'aide de \texttt{and others} ;
	\item \textbf{Ouvrage en langue étrangère}, pour lequel le titre n'est pas traduit,
	      les autres éléments bibliographiques restant conformes aux normes françaises ;
	\item \textbf{Livre en collection}, identifié par la présence des champs
	      \texttt{series} et, le cas échéant, \texttt{number}.
\end{itemize}

Les contributions insérées dans des ouvrages collectifs relèvent du type
\texttt{@incollection} et font l'objet d'une section distincte.

\paragraph{Locutions latines utilisées dans les citations}

Dans les démonstrations relatives aux ouvrages (livres), certaines locutions
latines traditionnellement utilisées dans les références bibliographiques
peuvent apparaître afin d'abréger les citations successives. Leur emploi est
automatiquement géré par \texttt{biblatex} en fonction de la proximité et de la
nature des références appelées.

\begin{itemize}
	\item \emph{Ibid.} (\emph{ibidem}) est employé lorsque deux citations consécutives
	      renvoient à la même œuvre. La mention de pagination peut alors être ajoutée
	      ou modifiée (par exemple : \emph{ibid.}, p.~45) ;
	\item \emph{Id.} / \emph{Ead.} (\emph{idem}, \emph{eadem}) désignent le même auteur
	      que la référence précédente, lorsqu'il s'agit d'une œuvre différente ;
	\item \emph{Op. cit.} (\emph{opere citato}) est utilisé pour renvoyer à une œuvre
	      déjà citée antérieurement, lorsque la citation n'est plus immédiatement
	      consécutive ;
	\item \emph{Loc. cit.} (\emph{loco citato}) renvoie à un passage précis d'une œuvre
	      déjà citée, dans un contexte non consécutif.
\end{itemize}

\begin{center}
	\renewcommand{\arraystretch}{1.3}
	\begin{tabular}{|p{3cm}|p{6.5cm}|p{4.5cm}|}
		\hline
		\textbf{Locution}                              &
		\textbf{Condition d'apparition}                &
		\textbf{Activation requise / Remarques}          \\
		\hline

		\emph{Ibid.}                                   &
		Deux citations consécutives renvoyant à la \textbf{même œuvre}.
		La pagination peut être identique ou modifiée. &
		Activée par défaut dans les styles académiques français
		(\texttt{ibidtracker=true}).                     \\
		\hline

		\emph{Id.} / \emph{Ead.}                       &
		Deux citations consécutives renvoyant au \textbf{même auteur}
		mais à des œuvres différentes.                 &
		Activée par défaut (\texttt{idemtracker=true}).
		Dépend du style choisi.                          \\
		\hline

		\emph{Op. cit.}                                &
		Retour à une œuvre déjà citée après \textbf{interruption}
		par une autre référence.                       &
		Activée par défaut dans les styles classiques
		(\texttt{opcittracker=true}).                    \\
		\hline

		\emph{Loc. cit.}                               &
		Retour à la \textbf{même œuvre et au même passage}
		dans un contexte non consécutif.               &
		Désactivée par défaut.
		Nécessite \texttt{loccittracker=true}.
		Usage déconseillé en production.                 \\
		\hline
	\end{tabular}
\end{center}



\subsection{Livres (\texttt{@book})}

Structure :
\textsc{AUTEUR}, Prénom. Titre. Édition. Lieu de publication : éditeur, année. Pagination.

\paragraph{Modèle générique}

\begin{verbatim}
@book{CleUnique:YYYY,
  author     = {Nom, Prénom and Nom2, Prénom2},
  title      = {Titre de l'ouvrage},
  edition    = {2e éd.},
  publisher  = {Nom de l'éditeur},
  location   = {Ville},
  year       = {YYYY},
  pages      = {xxx p.},
  series     = {Nom de la collection},
  number     = {Numéro dans la collection},
  shorttitle = {Titre court (facultatif, recommandé)}
}
\end{verbatim}

\textbf{Champs obligatoires} :
\texttt{author}, \texttt{title}, \texttt{publisher}, \texttt{location}, \texttt{year}, \texttt{pages}.

\textbf{Champs optionnels} :
\texttt{edition} (si $\neq$ 1\textsuperscript{re}),
\texttt{series}/\texttt{number} (collection),
\texttt{shorttitle} (fortement recommandé pour les références fréquemment rappelées via op. cit.).

\bigskip


\paragraph{Tests fonctionnels (livres)}
Cette section teste :
\begin{itemize}
	\item le rendu AFNOR complet (\texttt{\textbackslash fullcite}) ;
	\item les appels successifs avec ibid. et pagination ;
	\item l'usage de op. cit. avec et sans ambiguïté ;
	\item la gestion des auteurs multiples et de \texttt{and others} ;
	\item les ouvrages en langue étrangère.
\end{itemize}

\subsubsection*{Livre — monographie avec mention d'édition}

\paragraph{Source bibliographique (Bib\LaTeX)}

\begin{verbatim}
@book{Boursin:1991,
  author    = {Boursin, Jean-Louis},
  title     = {Comprendre la statistique descriptive},
  edition   = {Nouv. éd.},
  publisher = {A. Colin},
  location  = {Paris},
  year      = {1991},
  pages     = {168 p.}
}
\end{verbatim}

\paragraph{Citation complète (\texttt{\textbackslash fullcite})}

\subparagraph{Sans pagination}

\begin{verbatim}
\fullcite{Boursin:1991}
\end{verbatim}

\begin{bibexample}
	\fullcite{Boursin:1991}
\end{bibexample}

\subparagraph{Avec pagination explicite}

\begin{verbatim}
\fullcite[p.45]{Boursin:1991}
\end{verbatim}

\begin{bibexample}
	\fullcite[p.45]{Boursin:1991}
\end{bibexample}

\subparagraph{Avec référence explicite}

\begin{verbatim}
\fullcite[Voir aussi][p.45]{Boursin:1991}
\end{verbatim}

\begin{bibexample}
	\fullcite[Voir aussi][p.45]{Boursin:1991}
\end{bibexample}

\paragraph{Citation simple (\texttt{\textbackslash cite})}

\subparagraph{Sans pagination}

\begin{verbatim}
\cite{Boursin:1991}
\end{verbatim}

\begin{bibexample}
	\cite{Boursin:1991}
\end{bibexample}

\subparagraph{Avec pagination explicite}

\begin{verbatim}
\cite[p.45]{Boursin:1991}
\end{verbatim}

\begin{bibexample}
	\cite[p.45]{Boursin:1991}
\end{bibexample}

\subparagraph{Avec référence explicite}

\begin{verbatim}
\cite[Voir aussi][p.45]{Boursin:1991}
\end{verbatim}

\begin{bibexample}
	\cite[Voir aussi][p.45]{Boursin:1991}
\end{bibexample}

\paragraph{Citation en note de bas de page}

\begin{itemize}
	\item Sans pagination :
	      \footciteexamplelocal{Boursin:1991}

	\item Avec pagination explicite :
	      \footciteexamplelocal[][p.45]{Boursin:1991}

	\item Avec référence explicite :
	      \footciteexamplelocal[Voir aussi][p.45]{Boursin:1991}
\end{itemize}

\subsubsection*{Livre sans mention d'édition explicite\footnote{\emph{L'absence de mention d'édition implique qu'il s'agit de la première édition ou que l'éditeur ne la précise pas explicitement.}}}

\paragraph{Source bibliographique (Bib\LaTeX)}

\begin{verbatim}
@book{Fitoussi:1995,
  author    = {Fitoussi, Jean-Paul},
  title     = {Le débat interdit : monnaie, Europe, pauvreté},
  publisher = {Arléa},
  location  = {Paris},
  year      = {1995},
  pages     = {318 p.}
}
\end{verbatim}

\paragraph{Citation complète (\texttt{\textbackslash fullcite})}

\subparagraph{Sans pagination}

\begin{verbatim}
\fullcite{Fitoussi:1995}
\end{verbatim}

\begin{bibexample}
	\fullcite{Fitoussi:1995}
\end{bibexample}

\subparagraph{Avec pagination explicite}

\begin{verbatim}
\fullcite[p.45]{Fitoussi:1995}
\end{verbatim}

\begin{bibexample}
	\fullcite[p.45]{Fitoussi:1995}
\end{bibexample}

\subparagraph{Avec référence explicite}

\begin{verbatim}
\fullcite[Voir aussi][p.45]{Fitoussi:1995}
\end{verbatim}

\begin{bibexample}
	\fullcite[Voir aussi][p.45]{Fitoussi:1995}
\end{bibexample}

\paragraph{Citation simple (\texttt{\textbackslash cite})}

\subparagraph{Sans pagination}

\begin{verbatim}
\cite{Fitoussi:1995}
\end{verbatim}

\begin{bibexample}
	\cite{Fitoussi:1995}
\end{bibexample}

\subparagraph{Avec pagination explicite}

\begin{verbatim}
\cite[p.45]{Fitoussi:1995}
\end{verbatim}

\begin{bibexample}
	\cite[p.45]{Fitoussi:1995}
\end{bibexample}

\subparagraph{Avec référence explicite}

\begin{verbatim}
\cite[Voir aussi][p.45]{Fitoussi:1995}
\end{verbatim}

\begin{bibexample}
	\cite[Voir aussi][p.45]{Fitoussi:1995}
\end{bibexample}

\paragraph{Citation en note de bas de page}

\begin{itemize}
	\item Sans pagination :
	      \footciteexamplelocal{Fitoussi:1995}

	\item Avec pagination explicite :
	      \footciteexamplelocal[][p.45]{Fitoussi:1995}

	\item Avec référence explicite :
	      \footciteexamplelocal[Voir aussi][p.45]{Fitoussi:1995}
\end{itemize}

\subsubsection*{Auteurs multiples et \texttt{and others}\footnote{\emph{Lorsque le nombre d'auteurs excède le seuil défini par le style bibliographique, l'utilisation de \texttt{and others} permet d'activer l'abréviation automatique (\emph{et al.}) dans les citations, tout en conservant la liste complète des auteurs dans la bibliographie.}
	}}

\paragraph{Source bibliographique (Bib\LaTeX)}

\begin{verbatim}
@book{Hermet:2000,
  author    = {Hermet, Guy and Badie, Bertrand
               and Birnbaum, Pierre and others},
  title     = {Dictionnaire de la science politique
               et des institutions politiques},
  publisher = {A. Colin},
  location  = {Paris},
  year      = {2000},
  pages     = {287 p.}
}
\end{verbatim}

\paragraph{Citation complète (\texttt{\textbackslash fullcite})}

\subparagraph{Sans pagination}

\begin{verbatim}
\fullcite{Hermet:2000}
\end{verbatim}

\begin{bibexample}
	\fullcite{Hermet:2000}
\end{bibexample}

\subparagraph{Avec pagination explicite}

\begin{verbatim}
\fullcite[p.45]{Hermet:2000}
\end{verbatim}

\begin{bibexample}
	\fullcite[p.45]{Hermet:2000}
\end{bibexample}

\subparagraph{Avec référence explicite}

\begin{verbatim}
\fullcite[Voir aussi][p.45]{Hermet:2000}
\end{verbatim}

\begin{bibexample}
	\fullcite[Voir aussi][p.45]{Hermet:2000}
\end{bibexample}

\paragraph{Citation simple (\texttt{\textbackslash cite})}

\subparagraph{Sans pagination}

\begin{verbatim}
\cite{Hermet:2000}
\end{verbatim}

\begin{bibexample}
	\cite{Hermet:2000}
\end{bibexample}

\subparagraph{Avec pagination explicite}

\begin{verbatim}
\cite[p.45]{Hermet:2000}
\end{verbatim}

\begin{bibexample}
	\cite[p.45]{Hermet:2000}
\end{bibexample}

\subparagraph{Avec référence explicite}

\begin{verbatim}
\cite[Voir aussi][p.45]{Hermet:2000}
\end{verbatim}

\begin{bibexample}
	\cite[Voir aussi][p.45]{Hermet:2000}
\end{bibexample}

\paragraph{Citation en note de bas de page}

\begin{itemize}
	\item Sans pagination :
	      \footciteexamplelocal{Hermet:2000}

	\item Avec pagination explicite :
	      \footciteexamplelocal[][p.45]{Hermet:2000}

	\item Avec référence explicite :
	      \footciteexamplelocal[Voir aussi][p.45]{Hermet:2000}
\end{itemize}

\subsubsection*{Ouvrage en langue étrangère\footnote{\emph{Pour les ouvrages en langue étrangère, la langue du titre n'est pas traduite. Les éléments de description bibliographique (lieu, éditeur, pagination) restent conformes aux normes françaises.}}}

\paragraph{Source bibliographique (Bib\LaTeX)}

\begin{verbatim}
@book{Lazar:1991,
  author    = {Lazar, Marc and Waller, Michael and Courtois, Stephane},
  title     = {Comrades and brothers :
               communism and trade unions in Europe},
  publisher = {Cass},
  location  = {London},
  year      = {1991},
  pages     = {VIII--204 p.}
}
\end{verbatim}

\paragraph{Citation complète (\texttt{\textbackslash fullcite})}

\subparagraph{Sans pagination}

\begin{verbatim}
\fullcite{Lazar:1991}
\end{verbatim}

\begin{bibexample}
	\fullcite{Lazar:1991}
\end{bibexample}

\subparagraph{Avec pagination explicite}

\begin{verbatim}
\fullcite[p.45]{Lazar:1991}
\end{verbatim}

\begin{bibexample}
	\fullcite[p.45]{Lazar:1991}
\end{bibexample}

\subparagraph{Avec référence explicite}

\begin{verbatim}
\fullcite[Voir aussi][p.45]{Lazar:1991}
\end{verbatim}

\begin{bibexample}
	\fullcite[Voir aussi][p.45]{Lazar:1991}
\end{bibexample}

\paragraph{Citation simple (\texttt{\textbackslash cite})}

\subparagraph{Sans pagination}

\begin{verbatim}
\cite{Lazar:1991}
\end{verbatim}

\begin{bibexample}
	\cite{Lazar:1991}
\end{bibexample}

\subparagraph{Avec pagination explicite}

\begin{verbatim}
\cite[p.45]{Lazar:1991}
\end{verbatim}

\begin{bibexample}
	\cite[p.45]{Lazar:1991}
\end{bibexample}

\subparagraph{Avec référence explicite}

\begin{verbatim}
\cite[Voir aussi][p.45]{Lazar:1991}
\end{verbatim}

\begin{bibexample}
	\cite[Voir aussi][p.45]{Lazar:1991}
\end{bibexample}

\paragraph{Citation en note de bas de page}

\begin{itemize}
	\item Sans pagination :
	      \footciteexamplelocal{Lazar:1991}

	\item Avec pagination explicite :
	      \footciteexamplelocal[][p.45]{Lazar:1991}

	\item Avec référence explicite :
	      \footciteexamplelocal[Voir aussi][p.45]{Lazar:1991}
\end{itemize}

\paragraph{Vérification de la génération des locutions latines}

Le présent paragraphe a pour objectif de vérifier le bon fonctionnement de la
génération automatique des locutions latines par \texttt{biblatex} dans le cas
des ouvrages. Contrairement aux démonstrations précédentes, les locutions
(\emph{ibid.}, \emph{op. cit.}, \emph{loc. cit.}) ne sont ici ni neutralisées ni
contournées. Les citations sont volontairement enchaînées afin de provoquer leur
apparition et d'en vérifier le comportement effectif.

\begin{refsection}

	Première citation complète d'un ouvrage :

	\fullcite{Boursin:1991}

	Citation consécutive du même ouvrage avec pagination modifiée
	(attendu : \emph{ibid.}) :

	\cite[p.~45]{Boursin:1991}

	Interruption par la citation d'un autre ouvrage :

	\cite[p.~12]{Fitoussi:1995}

	Retour à l'ouvrage initial avec pagination différente
	(attendu : \emph{op. cit.}) :

	\cite[p.~52]{Boursin:1991}

	Nouvelle interruption par une autre référence :

	\cite[p.~18]{Fitoussi:1995}

	Retour au même ouvrage et au même passage que précédemment
	(cas théorique de \emph{loc. cit.}) :

	\cite[p.~52]{Boursin:1991}

\end{refsection}

\medskip

\textit{Remarque méthodologique :} la locution loc. cit. est désactivée par
défaut dans la majorité des styles \texttt{biblatex}, conformément aux usages
académiques contemporains. En l'absence d'activation explicite du
\texttt{loccittracker}, ce dernier appel produira généralement
op. cit., éventuellement accompagné de la pagination.

\subsection{Rapports institutionnels (\texttt{@report})}

\emph{Les rapports institutionnels relèvent du type \texttt{@report}. Ils se distinguent des ouvrages par la prééminence de l'institution productrice sur l'auteur individuel, ainsi que par la présence fréquente d'un type, d'un numéro et d'une série.}

\emph{Dans le cas des rapports institutionnels, l'institution fait office d'auteur principal. L'utilisation des champs \texttt{shortinstitution} et \texttt{sortkey} permet d'assurer une forme de citation abrégée cohérente et un classement alphabétique stable dans la bibliographie.}


Structure : Institution/AUTEUR. Titre. Type/numéro. Institution ou organisation, année (série).

\paragraph{Source bibliographique (Bib\LaTeX)}

\begin{verbatim}
@report{Insee:2020,
  shortinstitution = {Insee},
  sortkey          = {Insee},
  title            = {Démographie - Population au début du mois - France
                      (inclus Mayotte à partir de 2014)},
  type             = {Rapport},
  number           = {001641607},
  institution      = {Institut National de la Statistique et des études économiques},
  year             = {2020},
  series           = {Séries chronologiques}
}
\end{verbatim}

\paragraph{Citation complète (\texttt{\textbackslash fullcite})}

\subparagraph{Sans pagination}

\begin{verbatim}
\fullcite{Insee:2020}
\end{verbatim}

\begin{bibexample}
	\fullcite{Insee:2020}
\end{bibexample}

\subparagraph{Avec pagination explicite}

\begin{verbatim}
\fullcite[p.~12]{Insee:2020}
\end{verbatim}

\begin{bibexample}
	\fullcite[p.~12]{Insee:2020}
\end{bibexample}

\subparagraph{Avec référence explicite}

\begin{verbatim}
\fullcite[Voir aussi][p.~12]{Insee:2020}
\end{verbatim}

\begin{bibexample}
	\fullcite[Voir aussi][p.~12]{Insee:2020}
\end{bibexample}

\paragraph{Citation simple (\texttt{\textbackslash cite})}

\subparagraph{Sans pagination}

\begin{verbatim}
\cite{Insee:2020}
\end{verbatim}

\begin{bibexample}
	\cite{Insee:2020}
\end{bibexample}

\subparagraph{Avec pagination explicite}

\begin{verbatim}
\cite[p.~12]{Insee:2020}
\end{verbatim}

\begin{bibexample}
	\cite[p.~12]{Insee:2020}
\end{bibexample}

\subparagraph{Avec référence explicite}

\begin{verbatim}
\cite[Voir aussi][p.~12]{Insee:2020}
\end{verbatim}

\begin{bibexample}
	\cite[Voir aussi][p.~12]{Insee:2020}
\end{bibexample}

\paragraph{Citation en note de bas de page}

\begin{itemize}
	\item Sans pagination :
	      \footciteexamplelocal{Insee:2020}

	\item Avec pagination explicite :
	      \footciteexamplelocal[][p.~12]{Insee:2020}

	\item Avec référence explicite :
	      \footciteexamplelocal[Voir aussi][p.~12]{Insee:2020}
\end{itemize}

\paragraph{Vérification de la génération des locutions latines pour les rapports institutionnels}

Le présent paragraphe a pour objectif de vérifier le bon fonctionnement de la
génération automatique des locutions latines par \texttt{biblatex} dans le cas
des rapports institutionnels. Contrairement aux démonstrations précédentes, les
locutions (ibid., op. cit., loc. cit.) ne sont ici ni
neutralisées ni contournées. Les citations sont volontairement enchaînées afin
de provoquer leur apparition et d'en vérifier le comportement effectif.

\begin{refsection}

	Première citation complète d'un rapport institutionnel :

	\fullcite{Insee:2020}

	Citation consécutive du même rapport avec pagination modifiée
	(attendu : ibid.) :

	\cite[p.~12]{Insee:2020}

	Interruption par la citation d'un autre rapport institutionnel
	(ou, à défaut, d'un ouvrage) :

	\cite[p.~45]{Fitoussi:1995}

	Retour au rapport initial après interruption
	(attendu : op. cit.) :

	\cite[p.~18]{Insee:2020}

	Nouvelle interruption par une autre référence :

	\cite[p.~52]{Fitoussi:1995}

	Retour au même rapport et au même passage que précédemment
	(cas théorique de loc. cit.) :

	\cite[p.~18]{Insee:2020}

\end{refsection}

\medskip

\textit{Remarque méthodologique :} comme pour les ouvrages, la locution
loc. cit. est désactivée par défaut dans la majorité des styles
\texttt{biblatex}. En l'absence d'activation explicite du
\texttt{loccittracker}, le dernier appel produira généralement
op. cit., éventuellement accompagné de la pagination.

\subsection{Chapitres d'ouvrage collectif (\texttt{@incollection})}

\emph{Les chapitres d'ouvrage correspondent à des contributions autonomes
	insérées dans un ouvrage collectif dirigé. Ils relèvent du type
	\texttt{@incollection} et se distinguent des livres par la présence d'un titre
	de chapitre et d'un directeur scientifique.}

\textbf{À tester (chapitres)} : vérifier la présence de \og In\fg{}, de l'éditeur scientifique suivi de \texttt{editortype} (\og dir.\fg{}), puis la pagination précédée de \og p.\fg{}

Structure : NOM, Prénom. Titre du chapitre In NOM, Prénom dir. Titre de l'ouvrage collectif. Lieu : éditeur, année. p. xx-yy.

\paragraph{Modèle générique}

\begin{verbatim}
@incollection{Cle:YYYY,
  author     = {Nom, Prénom},
  title      = {Titre du chapitre},
  editor     = {Nom, Prénom},
  editortype = {dir.},
  booktitle  = {Titre de l'ouvrage collectif},
  publisher  = {Éditeur},
  location   = {Lieu},
  year       = {YYYY},
  pages      = {xx-yy}
}
\end{verbatim}

\paragraph{Source bibliographique (Bib\LaTeX)}

\begin{verbatim}
@incollection{Hassenteufel:1993,
  author     = {Hassenteufel, Patrick},
  title      = {Les automnes infirmiers (1988-1992) : dynamiques d'une mobilisation},
  editor     = {Fillieule, Olivier},
  editortype = {dir.},
  booktitle  = {Sociologie de la protestation},
  publisher  = {l'Harmattan},
  location   = {Paris},
  year       = {1993},
  pages      = {93-120}
}
\end{verbatim}

\paragraph{Citation complète (\texttt{\textbackslash fullcite})}

\subparagraph{Sans pagination explicite}

\begin{verbatim}
\fullcite{Hassenteufel:1993}
\end{verbatim}

\begin{bibexample}
	\fullcite{Hassenteufel:1993}
\end{bibexample}

\subparagraph{Avec pagination explicite}

\begin{verbatim}
\fullcite[p.~95]{Hassenteufel:1993}
\end{verbatim}

\begin{bibexample}
	\fullcite[p.~95]{Hassenteufel:1993}
\end{bibexample}

\subparagraph{Avec référence explicite}

\begin{verbatim}
\fullcite[Voir aussi][p.~95]{Hassenteufel:1993}
\end{verbatim}

\begin{bibexample}
	\fullcite[Voir aussi][p.~95]{Hassenteufel:1993}
\end{bibexample}

\paragraph{Citation simple (\texttt{\textbackslash cite})}

\subparagraph{Sans pagination}

\begin{verbatim}
\cite{Hassenteufel:1993}
\end{verbatim}

\begin{bibexample}
	\cite{Hassenteufel:1993}
\end{bibexample}

\subparagraph{Avec pagination explicite}

\begin{verbatim}
\cite[p.~95]{Hassenteufel:1993}
\end{verbatim}

\begin{bibexample}
	\cite[p.~95]{Hassenteufel:1993}
\end{bibexample}

\subparagraph{Avec référence explicite}

\begin{verbatim}
\cite[Voir aussi][p.~95]{Hassenteufel:1993}
\end{verbatim}

\begin{bibexample}
	\cite[Voir aussi][p.~95]{Hassenteufel:1993}
\end{bibexample}

\paragraph{Citation en note de bas de page}

\begin{itemize}
	\item Sans pagination :
	      \footciteexamplelocal{Hassenteufel:1993}

	\item Avec pagination explicite :
	      \footciteexamplelocal[][p.~95]{Hassenteufel:1993}

	\item Avec référence explicite :
	      \footciteexamplelocal[Voir aussi][p.~95]{Hassenteufel:1993}
\end{itemize}

\paragraph{Vérification de la génération des locutions latines pour les chapitres d'ouvrage}

Le présent paragraphe vise à vérifier le bon fonctionnement de la génération
automatique des locutions latines par \texttt{biblatex} dans le cas des chapitres
d'ouvrage collectif. Les citations sont volontairement enchaînées afin de
provoquer l'apparition de ibid., op. cit. et, le cas échéant, loc. cit.

\begin{refsection}

	Première citation complète d'un chapitre :

	\fullcite{Hassenteufel:1993}

	Citation consécutive du même chapitre avec pagination modifiée
	(attendu : ibid.) :

	\cite[p.~95]{Hassenteufel:1993}

	Interruption par la citation d'un autre ouvrage ou chapitre :

	\cite[p.~45]{Fitoussi:1995}

	Retour au chapitre initial après interruption
	(attendu : op. cit.) :

	\cite[p.~97]{Hassenteufel:1993}

	Nouvelle interruption par une autre référence :

	\cite[p.~52]{Fitoussi:1995}

	Retour au même chapitre et au même passage que précédemment
	(cas théorique de loc. cit.) :

	\cite[p.~97]{Hassenteufel:1993}

\end{refsection}

\medskip

\textit{Remarque méthodologique :} comme pour les ouvrages et les rapports
institutionnels, la locution loc. cit. est désactivée par défaut dans la
plupart des styles \texttt{biblatex}. En l'absence d'activation explicite du
\texttt{loccittracker}, le dernier appel produira généralement
op. cit., éventuellement accompagné de la pagination.

\subsection{Articles de revue (\texttt{@article})}

Structure : NOM, Prénom. Titre de l'article. Titre de la revue, mois année, volume, numéro, pagination.

Le champ \texttt{monthtext} permet d'indiquer un mois, un intervalle de mois ou une saison complète.

\textbf{À tester (articles)} : contrôler l'affichage du \texttt{monthtext} (remplace mois+année), du volume/numéro et de la pagination. Modifier temporairement \texttt{monthtext} pour vérifier le rendu libre.

\subsubsection*{Article avec volume et numéro}

\paragraph{Source bibliographique (Bib\LaTeX)}

\begin{verbatim}
@article{Leca:2001,
  author       = {Leca, Jean},
  title        = {Les 50 ans de la RFSP : une relecture cavalière des débuts},
  journaltitle = {Revue française de science politique},
  monthtext    = {fév.-avr. 2001},
  year         = {2001},
  volume       = {LI},
  number       = {1-2},
  pages        = {5-17}
}
\end{verbatim}

\paragraph{Citation complète (\texttt{\textbackslash fullcite})}

\subparagraph{Sans pagination explicite}

\begin{verbatim}
\fullcite{Leca:2001}
\end{verbatim}

\begin{bibexample}
	\fullcite{Leca:2001}
\end{bibexample}

\subparagraph{Avec pagination explicite}

\begin{verbatim}
\fullcite[p.~6]{Leca:2001}
\end{verbatim}

\begin{bibexample}
	\fullcite[p.~6]{Leca:2001}
\end{bibexample}

\subparagraph{Avec référence explicite}

\begin{verbatim}
\fullcite[Voir aussi][p.~6]{Leca:2001}
\end{verbatim}

\begin{bibexample}
	\fullcite[Voir aussi][p.~6]{Leca:2001}
\end{bibexample}

\paragraph{Citation simple (\texttt{\textbackslash cite})}

\subparagraph{Sans pagination}

\begin{verbatim}
\cite{Leca:2001}
\end{verbatim}

\begin{bibexample}
	\cite{Leca:2001}
\end{bibexample}

\subparagraph{Avec pagination explicite}

\begin{verbatim}
\cite[p.~6]{Leca:2001}
\end{verbatim}

\begin{bibexample}
	\cite[p.~6]{Leca:2001}
\end{bibexample}

\subparagraph{Avec référence explicite}

\begin{verbatim}
\cite[Voir aussi][p.~6]{Leca:2001}
\end{verbatim}

\begin{bibexample}
	\cite[Voir aussi][p.~6]{Leca:2001}
\end{bibexample}

\paragraph{Citation en note de bas de page}

\begin{itemize}
	\item Sans pagination :
	      \footciteexamplelocal{Leca:2001}

	\item Avec pagination explicite :
	      \footciteexamplelocal[][p.~6]{Leca:2001}

	\item Avec référence explicite :
	      \footciteexamplelocal[Voir aussi][p.~6]{Leca:2001}
\end{itemize}

\subsubsection*{Article sans volume, numéros regroupés}

\paragraph{Source bibliographique (Bib\LaTeX)}

\begin{verbatim}
@article{Simeant:1993,
  author       = {Sim\'eant, Johanna},
  title        = {Violence d'un répertoire, les sans-papiers de la grève à la faim},
  journaltitle = {Cultures et conflits},
  monthtext    = {print.-été 1993},
  year         = {1993},
  number       = {9-10},
  pages        = {315-338}
}
\end{verbatim}

\paragraph{Citation complète (\texttt{\textbackslash fullcite})}

\subparagraph{Sans pagination explicite}

\begin{verbatim}
\fullcite{Simeant:1993}
\end{verbatim}

\begin{bibexample}
	\fullcite{Simeant:1993}
\end{bibexample}

\subparagraph{Avec pagination explicite}

\begin{verbatim}
\fullcite[p.~320]{Simeant:1993}
\end{verbatim}

\begin{bibexample}
	\fullcite[p.~320]{Simeant:1993}
\end{bibexample}

\subparagraph{Avec référence explicite}

\begin{verbatim}
\fullcite[Voir aussi][p.~320]{Simeant:1993}
\end{verbatim}

\begin{bibexample}
	\fullcite[Voir aussi][p.~320]{Simeant:1993}
\end{bibexample}

\paragraph{Citation simple (\texttt{\textbackslash cite})}

\subparagraph{Sans pagination}

\begin{verbatim}
\cite{Simeant:1993}
\end{verbatim}

\begin{bibexample}
	\cite{Simeant:1993}
\end{bibexample}

\subparagraph{Avec pagination explicite}

\begin{verbatim}
\cite[p.~320]{Simeant:1993}
\end{verbatim}

\begin{bibexample}
	\cite[p.~320]{Simeant:1993}
\end{bibexample}

\subparagraph{Avec référence explicite}

\begin{verbatim}
\cite[Voir aussi][p.~320]{Simeant:1993}
\end{verbatim}

\begin{bibexample}
	\cite[Voir aussi][p.~320]{Simeant:1993}
\end{bibexample}

\paragraph{Citation en note de bas de page}

\begin{itemize}
	\item Sans pagination :
	      \footciteexamplelocal{Simeant:1993}

	\item Avec pagination explicite :
	      \footciteexamplelocal[][p.~320]{Simeant:1993}

	\item Avec référence explicite :
	      \footciteexamplelocal[Voir aussi][p.~320]{Simeant:1993}
\end{itemize}

\paragraph{Vérification de la génération des locutions latines pour les articles}

Le présent paragraphe vise à vérifier la génération automatique des locutions
latines (ibid., op. cit., et le cas échéant loc. cit.) pour
les articles de revue. Les citations sont volontairement enchaînées afin de
provoquer leur apparition.

\begin{refsection}

	Première citation complète d'un article :

	\fullcite{Leca:2001}

	Citation consécutive du même article avec pagination modifiée
	(attendu : ibid.) :

	\cite[p.~6]{Leca:2001}

	Interruption par la citation d'un autre article :

	\cite[p.~320]{Simeant:1993}

	Retour au premier article après interruption
	(attendu : op. cit.) :

	\cite[p.~8]{Leca:2001}

	Nouvelle interruption :

	\cite[p.~322]{Simeant:1993}

	Retour au même article et au même passage que précédemment
	(cas théorique de loc. cit.) :

	\cite[p.~8]{Leca:2001}

\end{refsection}

\medskip

\textit{Remarque méthodologique :} la locution loc. cit. est généralement
désactivée par défaut dans la plupart des styles \texttt{biblatex}. En l'absence
d'activation explicite du \texttt{loccittracker}, le dernier appel produira
habituellement op. cit., éventuellement accompagné de la pagination.

\subsection{Thèses ou mémoires (\texttt{@thesis})}

Structure : NOM, Prénom. Titre. Niveau du travail : discipline :
lieu : établissement : année. Pagination.

\emph{Les thèses et mémoires constituent des travaux universitaires non publiés
	ou diffusés de manière limitée. Ils relèvent du type \texttt{@thesis}, pour lequel
	le niveau du diplôme et la discipline occupent une place centrale dans la
	référence bibliographique.}

\textbf{À tester (thèses)} : vérifier l'enchaînement type \textrightarrow{} discipline \textrightarrow{} ville \textrightarrow{} établissement \textrightarrow{} année, puis la pagination.

\paragraph{Source bibliographique (Bib\LaTeX)}

\begin{verbatim}
@thesis{Brice:2004,
  author      = {Brice, Catherine},
  title       = {La monarchie et la construction de l'identité nationale italienne :
                 1861-1911},
  type        = {Thèse d'\'Etat},
  discipline  = {Histoire},
  location    = {Paris},
  institution = {Institut d'études politiques},
  year        = {2004},
  pages       = {3 vol.}
}
\end{verbatim}

\paragraph{Citation complète (\texttt{\textbackslash fullcite})}

\subparagraph{Sans pagination explicite}

\begin{verbatim}
\fullcite{Brice:2004}
\end{verbatim}

\begin{bibexample}
	\fullcite{Brice:2004}
\end{bibexample}

\subparagraph{Avec pagination explicite}

\begin{verbatim}
\fullcite[p.~125]{Brice:2004}
\end{verbatim}

\begin{bibexample}
	\fullcite[p.~125]{Brice:2004}
\end{bibexample}

\subparagraph{Avec référence explicite}

\begin{verbatim}
\fullcite[Voir aussi][p.~125]{Brice:2004}
\end{verbatim}

\begin{bibexample}
	\fullcite[Voir aussi][p.~125]{Brice:2004}
\end{bibexample}

\paragraph{Citation simple (\texttt{\textbackslash cite})}

\subparagraph{Sans pagination}

\begin{verbatim}
\cite{Brice:2004}
\end{verbatim}

\begin{bibexample}
	\cite{Brice:2004}
\end{bibexample}

\subparagraph{Avec pagination explicite}

\begin{verbatim}
\cite[p.~125]{Brice:2004}
\end{verbatim}

\begin{bibexample}
	\cite[p.~125]{Brice:2004}
\end{bibexample}

\subparagraph{Avec référence explicite}

\begin{verbatim}
\cite[Voir aussi][p.~125]{Brice:2004}
\end{verbatim}

\begin{bibexample}
	\cite[Voir aussi][p.~125]{Brice:2004}
\end{bibexample}

\paragraph{Citation en note de bas de page}

\begin{itemize}
	\item Sans pagination :
	      \footciteexamplelocal{Brice:2004}

	\item Avec pagination explicite :
	      \footciteexamplelocal[][p.~125]{Brice:2004}

	\item Avec référence explicite :
	      \footciteexamplelocal[Voir aussi][p.~125]{Brice:2004}
\end{itemize}

\paragraph{Vérification de la génération des locutions latines pour les thèses et mémoires}

Le présent paragraphe vise à vérifier le bon fonctionnement de la génération
automatique des locutions latines par \texttt{biblatex} dans le cas des thèses
et mémoires. Les citations sont volontairement enchaînées afin de provoquer
l'apparition de ibid., op. cit. et, le cas échéant, loc. cit.

\begin{refsection}

	Première citation complète d'une thèse :

	\fullcite{Brice:2004}

	Citation consécutive de la même thèse avec pagination modifiée
	(attendu : ibid.) :

	\cite[p.~125]{Brice:2004}

	Interruption par la citation d'un autre travail :

	\cite[p.~320]{Leca:2001}

	Retour à la thèse initiale après interruption
	(attendu : op. cit.) :

	\cite[p.~210]{Brice:2004}

	Nouvelle interruption par une autre référence :

	\cite[p.~322]{Leca:2001}

	Retour à la même thèse et au même passage que précédemment
	(cas théorique de loc. cit.) :

	\cite[p.~210]{Brice:2004}

\end{refsection}

\medskip

\textit{Remarque méthodologique :} comme pour les autres types de documents, la
locution loc. cit. est désactivée par défaut dans la majorité des styles
\texttt{biblatex}. En l'absence d'activation explicite du
\texttt{loccittracker}, le dernier appel produira généralement
op. cit., éventuellement accompagné de la pagination.


\subsection{Films (\texttt{@movie})}

Structure : NOM, Prénom, réal. Titre [support]. Éditeur, année. Description matérielle, durée.

\emph{Les films et documents audiovisuels relèvent du type \texttt{@movie}. La
	référence met en avant le réalisateur (champ \texttt{director}) et précise le
	support (\texttt{medium}) ainsi que l'étendue matérielle (\texttt{extent}).}

\paragraph{Source bibliographique (Bib\LaTeX)}

\begin{verbatim}
@movie{Jaeggi:2003,
  director   = {Jaeggi, Danielle},
  author     = {Jaeggi, Danielle},
  title      = {A l'écoute de la terre : voyage scientifique au centre de la terre},
  medium     = {cassette vidéo},
  publisher  = {Alcome},
  year       = {2003},
  extent     = {1 cass. vidéo VHS, 52 min.}
}
\end{verbatim}

\paragraph{Citation complète (\texttt{\textbackslash fullcite})}

\subparagraph{Sans précision}

\begin{verbatim}
\fullcite{Jaeggi:2003}
\end{verbatim}

\begin{bibexample}
	\fullcite{Jaeggi:2003}
\end{bibexample}

\subparagraph{Avec minutage explicite}

\begin{verbatim}
\fullcite[00:12:30--00:14:10]{Jaeggi:2003}
\end{verbatim}

\begin{bibexample}
	\fullcite[00:12:30--00:14:10]{Jaeggi:2003}
\end{bibexample}

\subparagraph{Avec référence explicite}

\begin{verbatim}
\fullcite[Voir aussi][00:12:30--00:14:10]{Jaeggi:2003}
\end{verbatim}

\begin{bibexample}
	\fullcite[Voir aussi][00:12:30--00:14:10]{Jaeggi:2003}
\end{bibexample}

\paragraph{Citation simple (\texttt{\textbackslash cite})}

\subparagraph{Sans précision}

\begin{verbatim}
\cite{Jaeggi:2003}
\end{verbatim}

\begin{bibexample}
	\cite{Jaeggi:2003}
\end{bibexample}

\subparagraph{Avec minutage explicite}

\begin{verbatim}
\cite[00:12:30--00:14:10]{Jaeggi:2003}
\end{verbatim}

\begin{bibexample}
	\cite[00:12:30--00:14:10]{Jaeggi:2003}
\end{bibexample}

\subparagraph{Avec référence explicite}

\begin{verbatim}
\cite[Voir aussi][00:12:30--00:14:10]{Jaeggi:2003}
\end{verbatim}

\begin{bibexample}
	\cite[Voir aussi][00:12:30--00:14:10]{Jaeggi:2003}
\end{bibexample}

\paragraph{Citation en note de bas de page}

\begin{itemize}
	\item Sans précision :
	      \footciteexamplelocal{Jaeggi:2003}

	\item Avec minutage explicite :
	      \footciteexamplelocal[][00:12:30--00:14:10]{Jaeggi:2003}

	\item Avec référence explicite :
	      \footciteexamplelocal[Voir aussi][00:12:30--00:14:10]{Jaeggi:2003}
\end{itemize}

\paragraph{Vérification de la génération des locutions latines pour les films}

Le présent paragraphe vise à vérifier le bon fonctionnement de la génération
automatique des locutions latines par \texttt{biblatex} dans le cas des films.
Les citations sont volontairement enchaînées afin de provoquer l'apparition de
ibid., op. cit. et, le cas échéant, loc. cit.

\begin{refsection}

	Première citation complète d'un film :

	\fullcite{Jaeggi:2003}

	Citation consécutive du même film avec minutage modifié
	(attendu : ibid.) :

	\cite[00:12:30--00:14:10]{Jaeggi:2003}

	Interruption par la citation d'une autre référence :

	\cite[p.~12]{Fitoussi:1995}

	Retour au film initial après interruption
	(attendu : op. cit.) :

	\cite[00:20:00--00:21:10]{Jaeggi:2003}

	Nouvelle interruption :

	\cite[p.~18]{Fitoussi:1995}

	Retour au même film et au même minutage que précédemment
	(cas théorique de loc. cit.) :

	\cite[00:20:00--00:21:10]{Jaeggi:2003}

\end{refsection}

\medskip

\textit{Remarque méthodologique :} comme pour les autres types de documents, la
locution loc. cit. est désactivée par défaut dans la majorité des styles
\texttt{biblatex}. En l'absence d'activation explicite du
\texttt{loccittracker}, le dernier appel produira généralement
op. cit., éventuellement accompagné du minutage.

\subsection{Vidéos en ligne (\texttt{@online})}

Structure : AUTEUR ou institution. Titre [vidéo en ligne]. Nom de l'éditeur,
date de publication [consultée le date de consultation]. URL.

\emph{Les vidéos en ligne relèvent du type \texttt{@online}. Elles se distinguent
	des films par l'absence de support matériel et par la mention explicite de la
	plateforme de diffusion ainsi que des dates de publication et de consultation.
	Le champ \texttt{entrysubtype} permet d'adapter automatiquement le libellé
	affiché (par exemple \og vidéo en ligne \fg{} ou \og site web \fg{}).}

\paragraph{Source bibliographique (Bib\LaTeX)}

\begin{verbatim}
@online{SciencesPo:2017,
  entrysubtype = {video},
  institution  = {Sciences Po},
  title        = {Sciences Po passe le bac : épreuve d'histoire},
  platform     = {YouTube},
  pubdate      = {2017-06-12},
  urldate      = {2017-06-14},
  url          = {https://www.youtube.com/watch?v=ZvSVkNZzvx4&t}
}
\end{verbatim}

\paragraph{Citation complète (\texttt{\textbackslash fullcite})}

\subparagraph{Sans précision}

\begin{verbatim}
\fullcite{SciencesPo:2017}
\end{verbatim}

\begin{bibexample}
	\fullcite{SciencesPo:2017}
\end{bibexample}

\subparagraph{Avec minutage ou indication de séquence}

\begin{verbatim}
\fullcite[à partir de 02:15]{SciencesPo:2017}
\end{verbatim}

\begin{bibexample}
	\fullcite[à partir de 02:15]{SciencesPo:2017}
\end{bibexample}

\subparagraph{Avec référence explicite}

\begin{verbatim}
\fullcite[Voir aussi][à partir de 02:15]{SciencesPo:2017}
\end{verbatim}

\begin{bibexample}
	\fullcite[Voir aussi][à partir de 02:15]{SciencesPo:2017}
\end{bibexample}

\paragraph{Citation simple (\texttt{\textbackslash cite})}

\subparagraph{Sans précision}

\begin{verbatim}
\cite{SciencesPo:2017}
\end{verbatim}

\begin{bibexample}
	\cite{SciencesPo:2017}
\end{bibexample}

\subparagraph{Avec indication de séquence}

\begin{verbatim}
\cite[à partir de 02:15]{SciencesPo:2017}
\end{verbatim}

\begin{bibexample}
	\cite[à partir de 02:15]{SciencesPo:2017}
\end{bibexample}

\subparagraph{Avec référence explicite}

\begin{verbatim}
\cite[Voir aussi][à partir de 02:15]{SciencesPo:2017}
\end{verbatim}

\begin{bibexample}
	\cite[Voir aussi][à partir de 02:15]{SciencesPo:2017}
\end{bibexample}

\paragraph{Citation en note de bas de page}

\begin{itemize}
	\item Sans précision :
	      \footciteexamplelocal{SciencesPo:2017}

	\item Avec indication de séquence :
	      \footciteexamplelocal[][à partir de 02:15]{SciencesPo:2017}

	\item Avec référence explicite :
	      \footciteexamplelocal[Voir aussi][à partir de 02:15]{SciencesPo:2017}
\end{itemize}

\paragraph{Vérification de la génération des locutions latines pour les vidéos en ligne}

Le présent paragraphe vise à vérifier le bon fonctionnement de la génération
automatique des locutions latines par \texttt{biblatex} dans le cas des vidéos
en ligne. Les citations sont volontairement enchaînées afin de provoquer
l'apparition de ibid., op. cit. et, le cas échéant,
loc. cit.

\begin{refsection}

	Première citation complète d'une vidéo en ligne :

	\fullcite{SciencesPo:2017}

	Citation consécutive de la même vidéo avec indication de séquence
	(attendu : ibid.) :

	\cite[à partir de 02:15]{SciencesPo:2017}

	Interruption par la citation d'une autre référence :

	\cite[p.~12]{Fitoussi:1995}

	Retour à la vidéo initiale après interruption
	(attendu : op. cit.) :

	\cite[à partir de 04:10]{SciencesPo:2017}

	Nouvelle interruption :

	\cite[p.~18]{Fitoussi:1995}

	Retour à la même vidéo et à la même séquence que précédemment
	(cas théorique de loc. cit.) :

	\cite[à partir de 04:10]{SciencesPo:2017}

\end{refsection}

\medskip

\textit{Remarque méthodologique :} comme pour les autres types de documents, la
locution loc. cit. est désactivée par défaut dans la majorité des styles
\texttt{biblatex}. En l'absence d'activation explicite du
\texttt{loccittracker}, le dernier appel produira généralement
op. cit., éventuellement accompagné de l'indication de séquence.


\paragraph{Vérification des libellés pour les vidéos en ligne}

La modification du champ \texttt{entrysubtype} permet de faire varier
automatiquement le libellé affiché dans la référence.

\begin{itemize}
	\item \texttt{entrysubtype = \{video\}} :
	      affichage du libellé \og vidéo en ligne \fg{} ;
	\item \texttt{entrysubtype = \{website\}} :
	      affichage du libellé \og site web \fg{}.
\end{itemize}

Ce mécanisme permet d'utiliser le type \texttt{@online} aussi bien pour des
contenus audiovisuels que pour des pages web institutionnelles.

\subsection{sites web (\texttt{@online})}

Structure : AUTEUR. Titre [en ligne]. Nom de l'éditeur, date de publication, date de mise à jour [date de consultation]. URL.

\emph{Les sites web relèvent du type \texttt{@online}. Ils se distinguent des
	vidéos en ligne par le libellé \og en ligne \fg{} et par la mise en avant de
	l'organisme éditeur plutôt que d'une plateforme de diffusion. Le champ
	\texttt{entrysubtype = \{website\}} permet d'activer automatiquement ce libellé.}

\textbf{À tester (en ligne)} : vérifier l'affichage de la plateforme (ou \texttt{journaltitle} pour un article), des dates \texttt{pubdate}/\texttt{updatedate} si présentes, de la date de consultation \texttt{urldate} entre crochets, puis l'URL. Tester \texttt{pubdate} aux formats \texttt{YYYY}, \texttt{YYYY-MM}, \texttt{YYYY-MM-DD} pour contrôler le rendu.

\paragraph{Source bibliographique (Bib\LaTeX)}

\begin{verbatim}
@online{MemoireDesHommes:2013,
  title        = {Mémoire des hommes},
  entrysubtype = {website},
  platform     = {France. Ministère de la défense},
  pubdate      = {2013},
  urldate      = {2017-06-19},
  url          = {http://www.memoiredeshommes.sga.defense.gouv.fr/}
}
\end{verbatim}

\paragraph{Citation complète (\texttt{\textbackslash fullcite})}

\subparagraph{Sans précision}

\begin{verbatim}
\fullcite{MemoireDesHommes:2013}
\end{verbatim}

\begin{bibexample}
	\fullcite{MemoireDesHommes:2013}
\end{bibexample}

\subparagraph{Avec précision contextuelle}

\begin{verbatim}
\fullcite[base de données nominative]{MemoireDesHommes:2013}
\end{verbatim}

\begin{bibexample}
	\fullcite[base de données nominative]{MemoireDesHommes:2013}
\end{bibexample}

\subparagraph{Avec référence explicite}

\begin{verbatim}
\fullcite[Voir aussi][base de données nominative]{MemoireDesHommes:2013}
\end{verbatim}

\begin{bibexample}
	\fullcite[Voir aussi][base de données nominative]{MemoireDesHommes:2013}
\end{bibexample}

\paragraph{Citation simple (\texttt{\textbackslash cite})}

\subparagraph{Sans précision}

\begin{verbatim}
\cite{MemoireDesHommes:2013}
\end{verbatim}

\begin{bibexample}
	\cite{MemoireDesHommes:2013}
\end{bibexample}

\subparagraph{Avec précision contextuelle}

\begin{verbatim}
\cite[base de données nominative]{MemoireDesHommes:2013}
\end{verbatim}

\begin{bibexample}
	\cite[base de données nominative]{MemoireDesHommes:2013}
\end{bibexample}

\subparagraph{Avec référence explicite}

\begin{verbatim}
\cite[Voir aussi][base de données nominative]{MemoireDesHommes:2013}
\end{verbatim}

\begin{bibexample}
	\cite[Voir aussi][base de données nominative]{MemoireDesHommes:2013}
\end{bibexample}

\paragraph{Citation en note de bas de page}

\begin{itemize}
	\item Sans précision :
	      \footciteexamplelocal{MemoireDesHommes:2013}

	\item Avec précision contextuelle :
	      \footciteexamplelocal[][base de données nominative]{MemoireDesHommes:2013}

	\item Avec référence explicite :
	      \footciteexamplelocal[Voir aussi][base de données nominative]{MemoireDesHommes:2013}
\end{itemize}

\paragraph{Vérification des libellés pour les sites web}

Le champ \texttt{entrysubtype = \{website\}} permet d'activer le libellé
\og en ligne \fg{} dans la référence bibliographique. L'organisme responsable
du site est indiqué à l'aide du champ \texttt{platform}, tandis que la date de
consultation est fournie par \texttt{urldate}.

La date de publication (\texttt{pubdate}) peut correspondre à l'année de mise
en ligne initiale ou, à défaut, à l'année de référence retenue par l'éditeur.

\paragraph{Vérification de la génération des locutions latines pour les sites web}

Le présent paragraphe vise à vérifier le bon fonctionnement de la génération
automatique des locutions latines par \texttt{biblatex} dans le cas des sites
web.

\textit{Remarque méthodologique :} pour les sites web, on conserve
généralement op. cit. La locution loc. cit. n'est pas retenue,
car il n'y a pas de pagination ni de repère stable comparable à une page.

\begin{refsection}

	Première citation complète d'un site web :

	\fullcite{MemoireDesHommes:2013}

	Citation consécutive du même site avec précision contextuelle
	(attendu : ibid.) :

	\cite[base de données nominative]{MemoireDesHommes:2013}

	Interruption par la citation d'une autre référence :

	\cite[p.~12]{Fitoussi:1995}

	Retour au site web initial après interruption
	(attendu : op. cit.) :

	\cite[base de données nominative]{MemoireDesHommes:2013}

\end{refsection}

\medskip

\subsection{Rapports institutionnels en ligne (\texttt{@online})}

Structure : Institution ou auteur institutionnel. Titre [en ligne].
Plateforme de diffusion, année de publication [date de consultation]. URL.

\emph{Les rapports institutionnels diffusés exclusivement en ligne relèvent du
	type \texttt{@online}. L'institution productrice fait office d'auteur, tandis
	que la plateforme de diffusion est indiquée séparément. Ce cas se distingue
	des rapports institutionnels classiques (\texttt{@report}) par l'absence de
	support éditorial autonome.}

\paragraph{Source bibliographique (Bib\LaTeX)}

\begin{verbatim}
@online{GlobalWebIndex:2020,
  author    = {{Global WebIndex}},
  title     = {COVID-19 level of concern by country worldwide 2020},
  platform  = {Statista},
  pubdate   = {2020},
  urldate   = {2020-09-01},
  url       = {https://www-statista-com.acces--distant.sciencespo.fr/statistics/1109124/covid-concern-level-by-country-worldwide/}
}
\end{verbatim}

\paragraph{Citation complète (\texttt{\textbackslash fullcite})}

\subparagraph{Sans précision}

\begin{verbatim}
\fullcite{GlobalWebIndex:2020}
\end{verbatim}

\begin{bibexample}
	\fullcite{GlobalWebIndex:2020}
\end{bibexample}

\subparagraph{Avec précision contextuelle}

\begin{verbatim}
\fullcite[données comparatives internationales]{GlobalWebIndex:2020}
\end{verbatim}

\begin{bibexample}
	\fullcite[données comparatives internationales]{GlobalWebIndex:2020}
\end{bibexample}

\subparagraph{Avec référence explicite}

\begin{verbatim}
\fullcite[Voir aussi][données comparatives internationales]{GlobalWebIndex:2020}
\end{verbatim}

\begin{bibexample}
	\fullcite[Voir aussi][données comparatives internationales]{GlobalWebIndex:2020}
\end{bibexample}

\paragraph{Citation simple (\texttt{\textbackslash cite})}

\subparagraph{Sans précision}

\begin{verbatim}
\cite{GlobalWebIndex:2020}
\end{verbatim}

\begin{bibexample}
	\cite{GlobalWebIndex:2020}
\end{bibexample}

\subparagraph{Avec précision contextuelle}

\begin{verbatim}
\cite[données comparatives internationales]{GlobalWebIndex:2020}
\end{verbatim}

\begin{bibexample}
	\cite[données comparatives internationales]{GlobalWebIndex:2020}
\end{bibexample}

\subparagraph{Avec référence explicite}

\begin{verbatim}
\cite[Voir aussi][données comparatives internationales]{GlobalWebIndex:2020}
\end{verbatim}

\begin{bibexample}
	\cite[Voir aussi][données comparatives internationales]{GlobalWebIndex:2020}
\end{bibexample}

\paragraph{Citation en note de bas de page}

\begin{itemize}
	\item Sans précision :
	      \footciteexamplelocal{GlobalWebIndex:2020}

	\item Avec précision contextuelle :
	      \footciteexamplelocal[][données comparatives internationales]{GlobalWebIndex:2020}

	\item Avec référence explicite :
	      \footciteexamplelocal[Voir aussi][données comparatives internationales]{GlobalWebIndex:2020}
\end{itemize}

\paragraph{Vérification de la génération des locutions latines pour les rapports institutionnels en ligne}

Le présent paragraphe vise à vérifier le bon fonctionnement de la génération
automatique des locutions latines par \texttt{biblatex} dans le cas des rapports
institutionnels diffusés en ligne. Les citations sont volontairement enchaînées
afin de provoquer l'apparition de ibid., op. cit. et, le cas
échéant, loc. cit.

\begin{refsection}

	Première citation complète d'un rapport institutionnel en ligne :

	\fullcite{GlobalWebIndex:2020}

	Citation consécutive du même rapport avec précision contextuelle
	(attendu : ibid.) :

	\cite{GlobalWebIndex:2020}

	Interruption par la citation d'une autre référence :

	\cite[p.~12]{Fitoussi:1995}

	Retour au rapport institutionnel initial après interruption
	(attendu : op. cit.) :

	\cite[p.~12]{GlobalWebIndex:2020}

	Nouvelle interruption :

	\cite[p.~18]{Fitoussi:1995}

	Retour au même rapport institutionnel et au même contexte que précédemment
	(cas théorique de loc. cit.) :

	\cite[p.~12]{GlobalWebIndex:2020}

\end{refsection}

\medskip

\textit{Remarque méthodologique :} comme pour les autres ressources en ligne, la
locution loc. cit. est désactivée par défaut dans la majorité des styles
\texttt{biblatex}. En l'absence d'activation explicite du
\texttt{loccittracker}, le dernier appel produira généralement
op. cit., sans modification du libellé.


\subsection{Articles de revue consultés en ligne (\texttt{@online})}

Structure : NOM, Prénom. Titre de l'article. Titre de la revue, volume,
numéro, pagination, date de publication [date de consultation]. Disponibilité
et accès.

\emph{Les articles de revue consultés exclusivement en ligne relèvent du type
	\texttt{@online} lorsque l'accès se fait via une plateforme numérique
	(JSTOR, Cairn, Persée, etc.). Ils conservent toutefois les éléments
	structurants d'un article de revue (\texttt{journaltitle}, volume, numéro,
	pagination).}

\paragraph{Source bibliographique (Bib\LaTeX)}

\begin{verbatim}
@online{RindeRokkan:1959,
  author       = {Rinde, Erik and Rokkan, Stein},
  title        = {Toward an International program of research on the handling of conflicts:
                   Introduction},
  journaltitle = {The Journal of Conflict Resolution},
  pubdate      = {1959-03},
  volume       = {3},
  number       = {1},
  pages        = {1-5},
  urldate      = {2005-04-12},
  url          = {http://www.jstor.org}
}
\end{verbatim}

\paragraph{Citation complète (\texttt{\textbackslash fullcite})}

\subparagraph{Sans précision}

\begin{verbatim}
\fullcite{RindeRokkan:1959}
\end{verbatim}

\begin{bibexample}
	\fullcite{RindeRokkan:1959}
\end{bibexample}

\subparagraph{Avec pagination explicite}

\begin{verbatim}
\fullcite[p.~2]{RindeRokkan:1959}
\end{verbatim}

\begin{bibexample}
	\fullcite[p.~2]{RindeRokkan:1959}
\end{bibexample}

\subparagraph{Avec référence explicite}

\begin{verbatim}
\fullcite[Voir aussi][p.~2]{RindeRokkan:1959}
\end{verbatim}

\begin{bibexample}
	\fullcite[Voir aussi][p.~2]{RindeRokkan:1959}
\end{bibexample}

\paragraph{Citation simple (\texttt{\textbackslash cite})}

\subparagraph{Sans pagination}

\begin{verbatim}
\cite{RindeRokkan:1959}
\end{verbatim}

\begin{bibexample}
	\cite{RindeRokkan:1959}
\end{bibexample}

\subparagraph{Avec pagination explicite}

\begin{verbatim}
\cite[p.~2]{RindeRokkan:1959}
\end{verbatim}

\begin{bibexample}
	\cite[p.~2]{RindeRokkan:1959}
\end{bibexample}

\subparagraph{Avec référence explicite}

\begin{verbatim}
\cite[Voir aussi][p.~2]{RindeRokkan:1959}
\end{verbatim}

\begin{bibexample}
	\cite[Voir aussi][p.~2]{RindeRokkan:1959}
\end{bibexample}

\paragraph{Citation en note de bas de page}

\begin{itemize}
	\item Sans pagination :
	      \footciteexamplelocal{RindeRokkan:1959}

	\item Avec pagination explicite :
	      \footciteexamplelocal[][p.~2]{RindeRokkan:1959}

	\item Avec référence explicite :
	      \footciteexamplelocal[Voir aussi][p.~2]{RindeRokkan:1959}
\end{itemize}

\paragraph{Vérification de la génération des locutions latines pour les articles de revue en ligne}

Le présent paragraphe vise à vérifier le bon fonctionnement de la génération
automatique des locutions latines par \texttt{biblatex} dans le cas des articles
de revue consultés en ligne. Les citations sont volontairement enchaînées afin
de provoquer l'apparition de \emph{ibid.}, \emph{op. cit.} et, le cas échéant,
\emph{loc. cit.}

\begin{refsection}

	Première citation complète d'un article en ligne :

	\fullcite{RindeRokkan:1959}

	Citation consécutive du même article avec pagination modifiée
	(attendu : \emph{ibid.}) :

	\cite[p.~2]{RindeRokkan:1959}

	Interruption par la citation d'une autre référence :

	\cite[p.~45]{Fitoussi:1995}

	Retour à l'article initial après interruption
	(attendu : \emph{op. cit.}) :

	\cite[p.~3]{RindeRokkan:1959}

	Nouvelle interruption :

	\cite[p.~52]{Fitoussi:1995}

	Retour au même article et au même passage que précédemment
	(cas théorique de \emph{loc. cit.}) :

	\cite[p.~3]{RindeRokkan:1959}

\end{refsection}

\medskip

\textit{Remarque méthodologique :} comme pour les autres ressources en ligne,
la locution loc. cit. est désactivée par défaut dans la majorité des
styles \texttt{biblatex}. En l'absence d'activation explicite du
\texttt{loccittracker}, le dernier appel produira généralement
op. cit., éventuellement accompagné de la pagination.

\section{Compilation}

\subsection{Commande locale}
\begin{verbatim}
latexmk -pdf -xelatex main.tex
\end{verbatim}
Cette commande enchaîne XeLaTeX et \texttt{biber}, nettoie automatiquement et produit \texttt{build/main.pdf}.

\subsection{Avec Docker/Makefile}
\begin{verbatim}
make pdf        % lance docker compose run latex
make clean      % nettoie via latexmk -C
\end{verbatim}
Le \texttt{docker-compose.yml} monte automatiquement les fichiers de style dans \texttt{\$TEXMFHOME}.

\newpage
\ifdefined\demobib
  \nocite{*}
\fi
\printbibliography[heading=bibintoc, title=Bibliographie]

\end{document}
